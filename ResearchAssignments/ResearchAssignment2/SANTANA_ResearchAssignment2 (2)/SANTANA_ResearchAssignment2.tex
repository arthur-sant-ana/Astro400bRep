%% Beginning of file 'sample7.tex'
%%
%% Version 7. Created January 2025.  
%%
%% AASTeX v7 calls the following external packages:
%% times, hyperref, ifthen, hyphens, longtable, xcolor, 
%% bookmarks, array, rotating, ulem, and lineno 
%%
%% RevTeX is no longer used in AASTeX v7.
%%
\documentclass[linenumbers,trackchanges]{aastex7}
%%
%% This initial command takes arguments that can be used to easily modify 
%% the output of the compiled manuscript. Any combination of arguments can be 
%% invoked like this:
%%
%% \documentclass[argument1,argument2,argument3,...]{aastex7}
%%
%% Six of the arguments are typestting options. They are:
%%
%%  twocolumn   : two text columns, 10 point font, single spaced article.
%%                This is the most compact and represent the final published
%%                derived PDF copy of the accepted manuscript from the publisher
%%  default     : one text column, 10 point font, single spaced (default).
%%  manuscript  : one text column, 12 point font, double spaced article.
%%  preprint    : one text column, 12 point font, single spaced article.  
%%  preprint2   : two text columns, 12 point font, single spaced article.
%%  modern      : a stylish, single text column, 12 point font, article with
%% 		  wider left and right margins. This uses the Daniel
%% 		  Foreman-Mackey and David Hogg design.
%%
%% Note that you can submit to the AAS Journals in any of these 6 styles.
%%
%% There are other optional arguments one can invoke to allow other stylistic
%% actions. The available options are:
%%
%%   astrosymb    : Loads Astrosymb font and define \astrocommands. 
%%   tighten      : Makes baselineskip slightly smaller, only works with 
%%                  the twocolumn substyle.
%%   times        : uses times font instead of the default.
%%   linenumbers  : turn on linenumbering. Note this is mandatory for AAS
%%                  Journal submissions and revisions.
%%   trackchanges : Shows added text in bold.
%%   longauthor   : Do not use the more compressed footnote style (default) for 
%%                  the author/collaboration/affiliations. Instead print all
%%                  affiliation information after each name. Creates a much 
%%                  longer author list but may be desirable for short 
%%                  author papers.
%% twocolappendix : make 2 column appendix.
%%   anonymous    : Do not show the authors, affiliations, acknowledgments,
%%                  and author contributions for dual anonymous review.
%%  resetfootnote : Reset footnotes to 1 in the body of the manuscript.
%%                  Useful when there are a lot of authors and affiliations
%%		    in the front matter.
%%   longbib      : Print article titles in the references. This option
%% 		    is mandatory for PSJ manuscripts.
%%
%% Since v6, AASTeX has included \hyperref support. While we have built in 
%% specific %% defaults into the classfile you can manually override them 
%% with the \hypersetup command. For example,
%%
%% \hypersetup{linkcolor=red,citecolor=green,filecolor=cyan,urlcolor=magenta}
%%
%% will change the color of the internal links to red, the links to the
%% bibliography to green, the file links to cyan, and the external links to
%% magenta. Additional information on \hyperref options can be found here:
%% https://www.tug.org/applications/hyperref/manual.html#x1-40003
%%
%% The "bookmarks" has been changed to "true" in hyperref
%% to improve the accessibility of the compiled pdf file.
%%
%% If you want to create your own macros, you can do so
%% using \newcommand. Your macros should appear before
%% the \begin{document} command.
%%
\newcommand{\vdag}{(v)^\dagger}
\newcommand\aastex{AAS\TeX}
\newcommand\latex{La\TeX}
%%%%%%%%%%%%%%%%%%%%%%%%%%%%%%%%%%%%%%%%%%%%%%%%%%%%%%%%%%%%%%%%%%%%%%%%%%%%%%%%
%%
%% The following section outlines numerous optional output that
%% can be displayed in the front matter or as running meta-data.
%%
%% Running header information. A short title on odd pages and 
%% short author list on even pages. Note that this
%% information may be modified in production.
%%\shorttitle{AASTeX v7 Sample article}
%%\shortauthors{The Terra Mater collaboration}
%%
%% Include dates for submitted, revised, and accepted.
%%\received{February 1, 2025}
%%\revised{March 1, 2025}
%%\accepted{\today}
%%
%% Indicate AAS Journal the manuscript was submitted to.
%%\submitjournal{PSJ}
%% Note that this command adds "Submitted to " the argument.
%%
%% You can add a light gray and diagonal water-mark to the first page 
%% with this command:
%% \watermark{text}
%% where "text", e.g. DRAFT, is the text to appear.  If the text is 
%% long you can control the water-mark size with:
%% \setwatermarkfontsize{dimension}
%% where dimension is any recognized LaTeX dimension, e.g. pt, in, etc.
%%%%%%%%%%%%%%%%%%%%%%%%%%%%%%%%%%%%%%%%%%%%%%%%%%%%%%%%%%%%%%%%%%%%%%%%%%%%%%%%
%%
%% Use this command to indicate a subdirectory where figures are located.
%%\graphicspath{{./}{figures/}}
%% This is the end of the preamble.  Indicate the beginning of the
%% manuscript itself with \begin{document}.

\begin{document}

\title{Charting M33's Stellar Stream Dynamics Following The Milky Way-M31 Merger}

%% A significant change from AASTeX v6+ is in the author blocks. Now an email
%% address is required for each author. This means that each author requires
%% at least one of the following:
%%
%% \author
%% \affiliation
%% \email
%%
%% If these three commands are not available for each author, the latex
%% compiler will issue an error and if you force the latex compiler to continue,
%% it will generate an incomplete pdf.
%%
%% Multiple \affiliation commands are allowed and authors can also include
%% an optional \altaffiliation to indicate a status, i.e. Hubble Fellow. 
%% while affiliations are indexed as footnotes, altaffiliations are noted with
%% with a non-numeric footnote that is set away from the numeric \affiliation 
%% footnotes. NOTE that if an \altaffiliation command is used it must 
%% come BEFORE the \affiliation call, right after the \author command, in 
%% order to place the footnotes in the proper location. Because non-numeric
%% symbols are used, \altaffiliation should be used sparingly.
%%
%% In v7 the \author command takes an optional argument which provides 
%% additional metadata about the author. Authors can provide the 16 digit 
%% ORCID, the surname (family or last) name, the given (first or fore-) name, 
%% and a name suffix, e.g. "Jr.". The syntax is:
%%
%% \author[orcid=0000-0002-9072-1121,gname=Gregory,sname=Schwarz]{Greg Schwarz}
%%
%% This name metadata in not shown, it is only for parsing by the peer review
%% system so authors can be more easily identified. This name information will
%% also be sent to the publisher so they can include it in the CROSSREF 
%% metadata. Including an orcid will hyperlink the author name to the 
%% author's ORCID page. Note that  during compilation, LaTeX will do some 
%% limited checking of the format of the ID to make sure it is valid. If 
%% the "orcid-ID.png" image file is  present or in the LaTeX pathway, the 
%% ORCID icon will appear next to the authors name.
%%
%% Even though emails are now required for each author, the \email does not
%% produce output in the compiled manuscript unless the optional "show" command
%% is used. For example,
%%
%% \email[show]{greg.schwarz@aas.org}
%%
%% All "shown" emails are show in the bottom left of the first page. Due to
%% space constraints, only a few emails should be shown. 
%%
%% To identify a corresponding author, use the \correspondingauthor command.
%% The command appends "Corresponding Author: " to the argument it appears at
%% the bottom left of the first page like the output from \email. 

\author{Arthur Sant'Ana}
\affiliation{University of Arizona}
\email{assantana@arizona.edu}  


%% Use the \collaboration command to identify collaborations. This command
%% takes an optional argument that is either a number or the word "all"
%% which tells the compiler how many of the authors above the command to
%% show. For example "\collaboration[all]{(DELVE Collaboration)}" wil include
%% all the authors above this command.
%%
%% Mark off the abstract in the ``abstract'' environment. 
%\begin{abstract}

%This example manuscript is intended to serve as a tutorial and template for
%authors to use when writing their own AAS Journal articles. The manuscript
%includes a history of \aastex\ and documents the new features in the
%previous versions as well as the new features in version 7. This
%manuscript includes many figure and table examples to illustrate these new
%features.  Information on features not explicitly mentioned in the article
%can be viewed in the manuscript comments or more extensive online
%documentation. Authors are welcome replace the text, tables, figures, and
%bibliography with their own and submit the resulting manuscript to the AAS
%Journals peer review system.  The first lesson in the tutorial is to remind
%authors that the AAS Journals, the Astrophysical Journal (ApJ), the
%Astrophysical Journal Letters (ApJL), the Astronomical Journal (AJ), and 
%the Planetary Science Journal (PSJ) all have a 250 word limit for the 
%abstract. The limit is 150 for RNAAS manuscripts. If you exceed this length 
%the Editorial office will ask you to shorten it. This abstract has 189 words.

%\end{abstract}

%% Keywords should appear after the \end{abstract} command. 
%% The AAS Journals now uses Unified Astronomy Thesaurus (UAT) concepts:
%% https://astrothesaurus.org
%% You will be asked to selected these concepts during the submission process
%% but this old "keyword" functionality is maintained in case authors want
%% to include these concepts in their preprints.
%%
%% You can use the \uat command to link your UAT concepts back its source.
%\keywords{\uat{Galaxies}{573} --- \uat{Cosmology}{343} --- \uat{High Energy astrophysics}{739} --- \uat{Interstellar medium}{847} --- \uat{Stellar astronomy}{1583} --- \uat{Solar physics}{1476}}

%% From the front matter, we move on to the body of the paper.
%% Sections are demarcated by \section and \subsection, respectively.
%% Observe the use of the LaTeX \label
%% command after the \subsection to give a symbolic KEY to the
%% subsection for cross-referencing in a \ref command.
%% You can use LaTeX's \ref and \label commands to keep track of
%% cross-references to sections, equations, tables, and figures.
%% That way, if you change the order of any elements, LaTeX will
%% automatically renumber them.

\section{Introduction} \label{sec:introduction}

For the most prominent model on how galaxies evolve, the standard $\Lambda$CDM cosmology, galaxies form and evolve via a series of mergers with neighboring smaller star systems \citep{Johnston2008-bz}. One of the predicted resulting factors in these mergers are coherent streams of stars that are stripped from their original host by the stronger tidal forces from the larger galaxy \citep{Jensen2021-mw}. As such, the existence of such streams hints at the scenario in which galaxies evolve via merging, and the dynamics of such objects may reveal crucial information on the history of a given galaxy.

Considering the distance these streams generally are from the galactic center, their dynamical timescales are much longer, which allows us to peer into past conditions of this system, forming something akin to a "fossil record" for past mergers of a specific galaxy \citep{Johnston1996-uv}. Thus, studying these streams' dynamics is deeply beneficial to understanding how galaxies evolve and can even provide insight on how different kinds of mergers affect the resulting galaxy, strengthening or weakening the $\Lambda$CDM framework. Furthermore, it has been discussed that measuring the amounts of stellar streams a galaxy has can put a lower limit to past merging events and also present indirect evidence for the existence of dark matter sub-haloes \citep{Malhan2018-mh}.

At this point in time, as \cite{Jensen2021-mw} evidences, we are getting access to revolutionary amounts of observational data concerning galactic structures, which includes cataloging Milky Way's stellar streams (Figure \ref{fig:general}). This revolution also comes in the form of deeply detailed simulations aiming at describing the behaviors of such streams and justifying the observations we are gathering \citep{Choi2007-lc}.

%% The "ht!" tells LaTeX to put the figure "here" first, at the "top" next
%% and to override the normal way of calculating a float position.
%% The asterisk after "figure" tells the compiler to span multiple columns
%% if a two column style is selected.
\begin{figure*}[ht!]
\epsscale{0.65}
\plotone{Stellarstream2.jpeg}
\caption{Chart demonstrating potential stream stars identified in \citep{Malhan2018-mh}. This evidences the astounding capability we have to identify these objects, however this only accounts for a small portion of all stream stars.}
\label{fig:general}
\end{figure*}

Despite the huge advances and breakthroughs seen in recent years, big questions remain. Some which only depend on our observational technology advancing, such as what the total quantity of stellar streams would be for our own galaxy, as even at this point some remain to be found \citep{Shipp2023-qz}. Others that, even with technological advancements, may still be impossible to answer completely, such as what the nature of Dark Matter is or how galaxies are created and evolve in a more detailed frame, given the presence of highly stochastic processes \citep{Amorisco2017-oc}. 




\section{Proposal} \label{sec:style}

\subsection{This Proposal}

The aim within this project is going to be to better describe the stellar streams that will originate from M33 during the Milky Way-M31 merger. More specifically I will be looking at the dynamic properties of such an event, including velocity gradients and dispersion in the stream. My final goal is to answer how these kinematics evolve throughout the merging process and whether or not we can draw any conclusions from its behavior that may advance current research. 

\subsection{Methods}

To achieve such goals many of the functions devised during classes and labs will be used. First the relevant snap shots from the simulation will be selected. In this case snap shots before and after each MW-M31 interaction will be the most useful, so around 3.8 Gyr ("M33$\_$260.txt"), 5.9 Gyr ({"M33$\_$420"}) and beyond 6.5 Gyr ("M33$\_$460"). Since stars are being addressed, only disk particles will be used. These points of interest can be clearly seen using the separation plot devised in Homework 6.
\begin{figure*}[h!]
\epsscale{0.7}
\plotone{MW_M31_M33_Sep.png}
\caption{Plot devised in Homework 6, which indicates the separation of objects thorughout the timeframe of the MW-M31 merger.}
\label{fig:general3}
\end{figure*}

Now we have to find which particles are not bound to M33 at each of these moments. In order to do that we will be using the "Jacobi Radius", which is defined as:

\begin{equation}
    R_j = r*{(\frac{M_{sat}}{2M_{host}(<r)})}^{\frac{1}{3}}
\end{equation}

This equation takes in the mass of the satellite M33 ($M_{sat}$), the mass of the host MW+M31 ($M_{host}$) and $r$, which is the separation between these two.

Both masses can be easily obtained directly from the "GalaxyMass.py" function created in Homework 3. The separation, however, will require some extra steps. The base code will come from Homework 6, where we used the "CenterOfMass" class and created the "orbitCOM" function in order to plot the separation between each object in terms of time. For this project we will be using "orbitCOM" in order to obtain the separation for a given time and plugging that into the Jacobi Radius formula.

Then finally we will be able to use the code developed on Lab 7 to make contour plots of the velocity gradients and color coded scatter plots for the velocity dispersion. The Jacobi radius will be used as a condition, so only particles beyond the Jacobi radius will be shown in the plots, allowing us to clearly see the characteristics and how they change from time to time during the merging event. See Figure \ref{fig:general2} for reference.

\begin{figure*}[h!]
\plotone{DiagramRA.png}
\caption{Diagram illustrating the steps that need to be taken in order to obtain velocity dispersion and gradient plots of particles located beyond the Jacobi radius (No longer bound by M33). In blue we can see more detailed explanations for how each function developed during an assignment will be used.
\label{fig:general2}}
\end{figure*}



\subsection{Hyposthesis}

I believe that, as the merging event advances, M33 will lose a large amount of the particles localized in its edges, most of it being during the major encounters of MW and M31, since its Jacobi radius will decrease since now both MW and M31 masses are exerting tidal forces. This will probably also mean we will see more extreme velocities beyond the Jacobi radius during those timeframes. Another prediction we can make is that as time passes these streams will slwoly be accreted by the now merged MW-M31 system, since this is what the $\Lambda$CDM model predicts.

%% For this sample we use BibTeX plus aasjournalv7.bst to generate the
%% the bibliography. The sample7.bib file was populated from ADS. To
%% get the citations to show in the compiled file do the following:
%%
%% pdflatex sample7.tex
%% bibtext sample7
%% pdflatex sample7.tex
%% pdflatex sample7.tex
\newpage
\bibliography{BibResearchAssi2}{}
\bibliographystyle{aasjournalv7}

%% This command is needed to show the entire author+affiliation list when
%% the collaboration and author truncation commands are used.  It has to
%% go at the end of the manuscript.
%\allauthors

%% Include this line if you are using the \added, \replaced, \deleted
%% commands to see a summary list of all changes at the end of the article.
%\listofchanges

\end{document}

% End of file `sample7.tex'.
